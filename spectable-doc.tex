\documentclass[12pt,a4paper]{scrreprt}
\usepackage[utf8]{inputenc}
\usepackage[T1]{fontenc}

\usepackage{graphicx}
\usepackage{url}
\usepackage{geometry}
\geometry{a4paper}


% uses the ``spectable'' package from spectable.sty
\usepackage{spectable}

\begin{document}
\pagestyle{empty}

\chapter{The ``spectable'' package}

The ``spectable'' package provides a simple way of tracking requirement specifications in a LaTeX document. Each requirement is given a unique ID based on the section it is defined in. It can be used to track the implementation of features in a project. The specification table below illustrates its usage.\\

This package was created for the sole purpose of simplifying my code in one of my projects -- it is not meant to be complete, correct or universally usable. Feel free to use it anywhere you like.

\section{Demo requirement table}

\begin{spectable}
\spec{The robot must have an emergency off button}{\valid}
\spec{The robot must have an on-off switch}{\new}
\spec{\strike{The robot must behave randomly}}{\deleted}
\spec{The robot must look cool}{\discussion{What exactly is meant by 'cool'?}}
\spec{The robot must be steam-powered}{\imposed}
\specinfo{A requirement ID consists of \texttt{Chapter.Section.Subsection.Sx}, where \texttt{x} is a unique number in this subsection}
\specinfo{If a specification claim becomes invalid, \textbf{don't} delete it, but rather mark it as \texttt{deleted} and \texttt{strike} the description.}
\end{spectable}

The code for the above table looks as follows:
\begin{verbatim}
\begin{spectable}
\spec{The robot must have an emergency off button}{\valid}
\spec{The robot must have an on-off switch}{\new}
\spec{\strike{The robot must behave randomly}}{\deleted}
\spec{The robot must look cool}{\discussion{What exactly is meant by 'cool'?}}
\spec{The robot must be steam-powered}{\imposed}
\specinfo{A requirement ID consists of \texttt{Chapter.Section.Subsection.Sx}, where \texttt{x} is a unique number in this subsection}
\specinfo{If a specification claim becomes invalid, \textbf{don't} delete it, but rather mark it as \texttt{deleted} and \texttt{strike} the description.}
\end{spectable}
\end{verbatim}

\section{License}

``spectable'' is licensed under the Creative Commons Attribution-ShareAlike 3.0 Unported License. To view a copy of this license, visit \url{http://creativecommons.org/licenses/by-sa/3.0/} or send a letter to Creative Commons, 444 Castro Street, Suite 900, Mountain View, California, 94041, USA. \\
\begin{center}
\includegraphics{by-sa.png}
\end{center}




\end{document}  